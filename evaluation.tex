\section{Evaluation}
\label{evaluation}
All tests were done on Tuxedo server which consists of 4 K80 GPUs and 2 GTX 1080 GPUs.
Among those devices, we used two GPUs within the single node.
We tested push-style, round based bfs of lonestardist. For input graphs, we used 
social network graph, twitter40i and livejournal, web-crawled graph, indochina-2004, and road network, road-USA. 
For each graph, we used nodes whose degree is maximum as the start node. Each test was done three times and 
calculated averages.

\begin{table}[!tb]
  \small
  \centering
  %\ra{1.3}
  \begin{tabular}{@{}p{60pt}p{35pt}p{35pt}p{35pt}p{35pt}@{}}
%\begin{tabularx}{50pt}{@{}lrp{2cm}@{}}    
    \toprule[1.2pt]
    Input Graphs   & Orig.   & Async. memcpy. & UVA & RDMA. \\
    \midrule
    livejournal    &  \textbf{0.02s}  & \textbf{0.02s}      & 0.81s   & 0.81s \\
    indochina-2004 &  0.005s & \textbf{0.003s}     & 0.93s   & 0.93s\\
    twitter40      &  0.6s   & \textbf{0.58s}      & 19.23s  & 20.3s\\
    road-USA       &  \textbf{1.8s}   & \textbf{1.8s}       & 12.5s   & 20s  \\
    \bottomrule[1.2pt]
   %\end{tabularx}
  \end{tabular}
  \vspace{5pt} \mycaption{Runtime with each technique}{}
  \label{tbl-op}
  \vspace{-15pt}
\end{table}


Table~\ref{tbl-op}t shows execution time (s) for each combination of input graphs and optimizations.
The first column shows original unoptimized bfs\_push results. From the second column to the last columns,
the results from the second column to the last columns respectively show 
only using asynchronous memory copy, only using UVA and RDMA with UVA and asynchronous memory copy. 
Among the techniques, asynchronous memory copy showed the best performance.

\subsection{Asynchronous memory copy}
This technique increases only 1 to 2 percentage from the original one.
It is expected results since we only target to overlap memory operations.
We expect that further overlapping, for example, between operator computation and memory operation, 
will show better results.

\subsection{UVA}
Interestingly, UVA technique decreases performance from the original bfs from about x10 to x100.
As we discussed in the previous sections, it implies that we need to consider two factors to use this technique.
First, the amount of pinned memory could be adjusted depending on the memory consumption.
It can cause frequent swapping of the memory which is not allocated to UVA and managed by kernel.
Second, frequent fetching remote memory could have critically impact on the performance.
We expect that considering data locality could improve performance. 
The above reasons are not verified with the experimental results even though they are obvious properties of UVA.
Further analysis and optimization could improve performance a lot.

\subsection{GPUDirect RDMA with UVA and asynchronous CUDA memory copy}
It shows the worst performance among other techniques.
But it is expected. First, we cannot fully exploit multiple physical nodes to 
optimize network overheads. The current version can support only one-to-one communication.
Second, the MPI communication is combination of non-blocking and blocking. However, the original Gluon 
uses fully asynchronous communications with the background specialized thread.
Devising new GPU vector and exploiting background worker thread as the original one does could solve the above two issues.
Third, it uses UVA. We showed that UVA without any optimization could ruin the performances.

To sum up, all three techniques do not achieve better performances than the original Gluon.
It implies that just applying technique could not improve performances a lot, but also there are many factors 
that we can consider and improve.
